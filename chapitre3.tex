%%**********************Partie 3 *********************** 


 \chapter{MISE EN PLACE D’UN RETOUR D’ÉTAT ET D’UN PRÉ-COMPENSATEUR}

L'asservissement de position du système $S_{v_{m}\longmapsto v_{s}}$ , est réalisé selon la loi de commande par retour d'état :\\
 $v_{m}(t)=Nk_{e}\theta_{r}(t)-Kx(t)=Nv_{r}(t)-Kx(t)$ , avec : $K=[k_{1} \hspace{2mm} k_{2}]$.\\\\
 
 \section{Calcul des paramètre du retour d'état dans la base initial :}
 
\subsection{Détermination de l'expression du système ($A_{bf}$, $B_{bf}$, $C_{bf}$ et $D_{bf}$) :}

D'après notre système on a l'expression :\\
\\\\
$\dot{x}=Ax+Bv_{m}=Ax+B(Nv_{r}-Kx)=(A-BK)x+BNv_{r}$
\\\\
On obtient alors notre matrice dynamique \quad $A_{bf}$=(A-BK)\\
\\
et : \quad $B_{bf}$=B.N \\
 et :\quad $C_{bf}$=[1 0] \\
 et :\quad $D_{bf}$=[0] \\\\\\\\\\

\subsection{L'expression de $A_{bf}$ en fonction de $[k_{1}$,$k_{2}]$ : }

On a $A_{bf}$=(A-BK),  avec: $K=[k_{1} \hspace{2mm} k_{2}]$
D'ou :\\\\
$A_{bf}=
\begin{bmatrix} 
0 & \frac{K_{s}}{9K_{g}} \\
-\frac{K_{g}K_{m}}{T_{m}}.k_{2} & -\frac{K_{g}K_{m}}{T_{m}}.k_{2}-\frac{1}{T_{m}}
\end{bmatrix}$ \\\\

\subsection{Justification la possibilité de placer les V.P de la matrice $A_{bf}$ et calcul de K :}

On a les valeurs propres : $v_{1},v_{2}=-2.4\pm5.5j$ \\
et:\\
$A_{bf}=
\begin{bmatrix} 
0 & \frac{K_{s}}{9K_{g}} \\
-\frac{K_{g}K_{m}}{T_{m}}.k_{2} & -\frac{K_{g}K_{m}}{T_{m}}.k_{2}-\frac{1}{T_{m}}
\end{bmatrix}$ \\\\

L'expression du polynôme caractéristique est:\\
P($\lambda$)=$\lambda^2+(\frac{1}{T_{m}}-\frac{K_{g}K_{m}}{T_{m}}.k_{2})\lambda+\frac{K_{s}}{9K_{g}}.\frac{K_{g}K_{m}}{T_{m}}.k_{1}$\\\\

On place les valeurs propres à $p_{1},p_{2}=-2.4\pm5.5j$\\

On obtient alors le polynôme suivant: \\\\
$(\lambda+2.4+5.5j)(\lambda+2.4-5.5j)={\lambda}^2+4.8\lambda+36.01$\\\\

Et par identification avec le polynôme P($\lambda$) et en tenant compte des valeurs expérimentales de $K_{g} K_{m} K_{s}$ et$T_{m}$ , on trouve alors:  $k_1$=1.389à \quad $k_2$=0.5986 \hyperref[section1.1]{(voir Annexe)}\label{annexe1}\\



\section{Calcul des paramètres du retour d'état en utilisant la forme compagne de commande :}
